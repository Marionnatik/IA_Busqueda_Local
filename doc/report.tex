\documentclass{article}

\setlength{\paperwidth}{210mm}
\setlength{\paperheight}{297mm}
\setlength{\hoffset}{-12mm}
\setlength{\voffset}{-10mm}
\setlength{\evensidemargin}{0mm}
\setlength{\oddsidemargin}{0mm}
\setlength{\topmargin}{0mm}
\setlength{\headheight}{0mm}
\setlength{\headsep}{5mm}
\setlength{\textheight}{256.2mm}
\setlength{\textwidth}{184.2mm}
\setlength{\marginparsep}{0mm}
\setlength{\marginparwidth}{0mm}
\setlength{\footskip}{5mm}
\setlength{\marginparpush}{0mm}


\usepackage[spanish]{babel}
\usepackage[utf8]{inputenc}

\pagestyle{myheadings}
\markright{IA - Práctica de búsqueda local}

\title{\Huge{Práctica de búsqueda local} \\
\vspace{15mm}
   \Large{Laboratorio de Inteligencia Artificial}}
\author{Marion Not - Michael Boris Mandirola}
\date{Primavera 2010-2011}

\begin{document}
\maketitle
\newpage
\tableofcontents
\newpage
\section*{Introducción}


\section{Representación del problema}


\subsection{Identificación}
%descripcion, analisis y caracteristicas del problema


\subsection{Estado}
\subsubsection{Elementos del estado}
%descripcion detallada y justificacion
\subsubsection{Tamaño del espacio de búsqueda}


\subsection{Operadores}
%descripcion detallada, condiciones de aplicacion, efecto
%factor de ramificacion
%justificacion de la eleccion
\subsubsection{Desplazamiento de una petición dentro de un centro}
\subsubsection{Intercanvio de capacidades de camiones}
\subsubsection{Operadores no implementados}


\subsection{Funciones Heurísticas}
%descripcion y analisis de los factores que intervienen en las heuristicas
%justificacion de la eleccion de estas heuristicas y de las ponderaciones
%explicacion del impacto de las heuristicas en la busqueda
\subsubsection{Maximización de la ganancia}
\subsubsection{Minimización de la diferencia entre hora de entrega deseada y
efectiva}


\subsection{Estados Iniciales}
%justificacion de la eleccion de los estados (bondad de la solucion, coste,
% adaptacion a cada algoritmo)
%descripcion de la implementacion

En un primer momento hemos intentado generar estados diferentes con lógicas muy diferentes con la idea de buscar nuestros estados iniciales entre un grupo más amplio y en particular elegir dos estados que tengan características muy diferentes. Es decir tiempo de generación de el estado inicial, complejidad de el algoritmo, posibilidades de evolución, calidad de la solución buscada.

\subsubsection{a}
A) Este estado divide los camiones de manera ecua entre los centres asignando los camiones con mas capacidad a los transportes mas tempranos. Las entregas están entregadas lo mas pronto posible: se hace una lista de peticiones pertinentes a un centro, se ordenan crecientemente por tiempo de entrega y disminuyendo por precio y se pasa toda la lista poniendo las peticiones en el transporte más pronto que tiene más tiempo libre.

\subsubsection{b}
B) Este estado ordena las peticiones por precio y hora. Por cada petición, intenta entregarla en la hora pertinente con este algoritmo:
Si no hay camión, se pone el camión disponible más pequeño y la petición, si hay camión y bastante capacidad libre también se pone la petición. Si hay camión de capacidad inferior a la capacidad máxima y hay disponibilidad de camiones más grandes, se asigna un camión mas grande en lugar de lo más pequeño y se pone la petición. Además, si no hay camiones libres más grandes intenta hacer lo mismo en las horas más tempranas hasta las 8. Si no tiene éxito, intenta hacerlo en las horas después hasta las 17.
Después si hay oras sin camión, se asignan los camiones que se quedan libres.

\subsubsection{f}
F) Los camiones se asignan como en el estado A. Sin ordinación se iteran todas las peticiones intentando ponerlas en su hora de pertinencia. Después, iterativamente por cada petición se intenta poner, las que se quedan no entregadas, en las horas más tempranas de la suya hasta las 8 y si no se tiene éxito, se intenta hacerlo en las horas después hasta las 17.

\subsubsection{v}
V) Los camiones se asignan como en el estado A. Las peticiones se quedan todas no entregadas.


\section{Implementación}

\subsection{Generación de problemas aleatorios}

\subsection{Uso del AIMA}

\subsection{Ejecucción de tests}
%input y script

\subsection{Recuperación de resultados}
%output


\section{Resultados}
% explicaciones y analisis detallada
% comparar los resultados con lo que se esperaba
% hacer los experimentos del enunciado y contestar las preguntas
% hacer tb mas experimentos para complementar

\subsection{Influencia de la solución inicial}
%coste de creacion
% influenca en la bondad de la solucion y en el coste del algoritmo

\subsection{Influencia de los operadores}
% comparacion entre conjuntos de operadores
% impacto sobre el coste temporal y bondad de la solucion

\subsection{Influencia de la heurística}
% comparacion entre las dos heuristicas, impacto en tiempo y bondad de solucion
% hacer experimentos para ver la influenca de las ponderaciones

\end{document}
